\documentclass{article}
\usepackage{amsthm}
\usepackage{amsmath}
\usepackage{amssymb}
\usepackage{parskip}
\usepackage{cleveref}
\usepackage{listings}
\usepackage[margin=1.25in]{geometry}
\newtheorem{thm}{Theorem}

\begin{document}
	\section{(U) Definitions}
		$\mathbb L$ denotes the set of all programming languages.

		For all $l \in \mathbb L$, $p(l)$ denotes the set of all computer
		programs which are written in $l$.

		$\mathit{ZZ}$ denotes the class of all sets.

		For all classes $G$, for all objects $x$, $G(x)$ iff $x$ belongs to
		$G$.
	\section{(U) Cross-Platform Stuff}
			Let $C$ denote the set of all cross-platform computer programs.
		\subsection{(U) Haskell}
			Let $h$ denote the Haskell programming language.

			$G$ denotes program 322, as contained in \cref{subsection:p322}.
			\begin{thm}
				$G \notin C$.
			\end{thm}
			\begin{proof}
				${}$

				For all computer programs $k$, $k \notin C$ if there exist
				operating systems $a$ and $b$ such that $k$ runs on $a$ and
				$k$ does not run on $b$.

				$G$ runs on OpenBSD, and $G$ does not run on Microsoft Windows.

				\[
					\therefore G \notin C.\qedhere
				\]
			\end{proof}
			\begin{thm}
				$G \in p(h) - C$.
			\end{thm}
			\begin{proof}
				\[
					\forall \left\{a,b,c\right\},\ 
					\mathit{ZZ}(a) \land \mathit{ZZ}(b) \land
					c \in a \land c \notin b \iff c \in a - b.
				\]
				\[
					\mathit{ZZ}\big(p(h)\big).
				\]
				\[
					\mathit{ZZ}(C).
				\]
				\[
					G \in p(h).
				\]
				\[
					G \notin C.
				\]
				\[
					\therefore G \in p(h) - C.\qedhere
				\]
			\end{proof}
			\begin{thm}
				$p(h) - C \neq \left\{\right\}$.
			\end{thm}
			\begin{proof}
				\[
					\forall x,\ 
					\mathit{ZZ}(x) \implies 
					\left(x \neq \left\{\right\} \iff \exists y \in x\right).
				\]
				\[
					\mathit{ZZ}\big(p(h) - C\big).
				\]
				\[
					G \in p(h) - C.
				\]
				\[
					\therefore p(h) - C \neq \left\{\right\}.\qedhere
				\]
			\end{proof}
		\subsection{(U) C}
			Let $h$ denote the C programming language.

			$G$ denotes program 2636, as contained in \cref{subsection:p2636}.
			\begin{thm}
				$p(h) \cap C \neq \left\{\right\}$.
			\end{thm}
			\begin{proof}
				\[
					\forall x,\ 
					\mathit{ZZ}(x) \implies 
					\left(x \neq \left\{\right\} \iff \exists y \in x\right).
				\]
				\[
					\mathit{ZZ}\big(p(h) \cap C\big).
				\]
				\[
					G \in p(h) \cap C.
				\]
				\[
					\therefore p(h) \cap C \neq \left\{\right\}.\qedhere
				\]
			\end{proof}
	\section{(U) Programs}
		\subsection{(U) 322}\label{subsection:p322}
			\begin{lstlisting}[language=Haskell]
import System.Environment;

main :: IO ();
main = getEnvironment >>= putStrLn . thePWD
  where thePWD = (!!0) . filter ((== "PWD") . fst);
			\end{lstlisting}
		\subsection{(U) 2636}\label{subsection:p2636}
			\begin{lstlisting}[language=C]
#include <stdio.h>
int main()
{
	printf("Hello, world.\n");
	return 0;
}
			\end{lstlisting}
\end{document}
